\section*{Integration Untermannigfaltigkeiten}

\begin{karte}{Zerlegung der Eins}
    Zerlegung der Eins: 
    \[ \abb{h_\alpha}{M}{\R}, p \mapsto \frac{\mathds{1}_{\varphi_\alpha(U_\alpha)}(p) }{ \sum_{\beta \in A} \mathds{1}_{ \varphi_\beta (U_\beta) }(p) }, \quad p\in M \]

    Es seien 
    \begin{enumerate}
        \item \( \zeta_k(M) := \zeta_k = \set{ E \subset M: (\varphi_\alpha^{-1}((E\cap \varphi_\alpha(U_\alpha))) \text{ Borelmessbare Menge in } \R^k } \)
        \item \( S_M^k := S^k(E) = \sum_{\alpha \in A} \int_{U_\alpha} (h_\alpha \mathds{1}_E) 
        \circ \varphi_\alpha(\det D \varphi_\alpha^t D \varphi_\alpha)^{1/2} \dx{\lambda^k} \).
    \end{enumerate}

    Es gilt 
    \begin{enumerate}
        \item \( \zeta_k(M) \) ist eine \(\sigma\)-Algebra.
        \item \( \abb{S_M^k}{\zeta_k(M)}{[0,\infty]} \) ist ein Maß.
        \item Die Definition von \( \zeta_k(M) \) und \( S_M^k \) ist unabhängig von der Wahl des Atlas'.
    \end{enumerate}
\end{karte}

\begin{karte}{Integral bzgl. \( \zeta_k \) und \( S_M^k \)}
    \( \abb{f}{M}{\bar{\R}} \) ist integrierbar über \(M\), falls 
    \[ \int_M \abs{f} \dx{S_M^k} < \infty. \]

    Das Integral ist dann definiert als 
    \[ \int_M f \dx{S_M^k} := \sum_{\alpha\in A} \int_{U_\alpha} f(\varphi_\alpha(t)) \cdot (\det (D^T_{\varphi_\alpha}  D_{\varphi_\alpha}))^{1/2} \lambda^k(\mathrm{d}t). \]
\end{karte}

\begin{karte}{Ähnlichkeitssatz}
    Sei \( \abb{T}{\R^n}{\R^n} \) eine Ähnlichkeit der Form 
    \( T(p) = \lambda Gp + a, a \in \R^n, G\in \mathrm{O}(n), \lambda > 0 \). Ist \(M \subset \R^n \)
    eine \(\mathcal{C}^1\)-Untermannigfaltigkeit, so ist auch \(N := T(M) \) eine Untermannigfaltigkeit 
    und für die zugehörigen Maße gilt: 
    \begin{enumerate}
        \item Ist \(A \subset M\) messbar (\(A\in \zeta_k(M)\)), so ist \(T(A) \in \zeta_k(N)\) 
        mit \( S_n^k(T(A)) = \lambda^k S_m^k(A) \).
        \item Für positive messbare oder integrierbare Funktionen \( \abb{f}{N}{\bar{R}} \) gilt 
        \[ \int_N f \dx{S_N^k} = \lambda^k \int_M f \circ T \dx{S_M^k}. \]
    \end{enumerate}
\end{karte}

\begin{karte}{Zwiebelformel}
    Sei \( f\in \mathcal{L}^1(\R^n) \). Dann folgt \( f|_{\partial B_r} \in \mathcal{L}^1(S_{\partial B}^{n-1}), 
    B_r = B_r(0), \partial B_r = \partial B_r(0) \).

    \begin{align*}
        \int_{R^n} f \dx{\lambda} &= \int_0^\infty \left( \int_{\partial B_r} f \dx{S_{\partial B_r}^{n-1}} \right) \dx{r} \\
        &= \inf_0^\infty \int_{\mathcal{B}^{n-1}} f(r w) S_{\mathcal{B}^{n-1}}^{n-1} \dx{w} \dx{r}.
    \end{align*} 

\end{karte}

\begin{karte}{Regulärer Rand}
    Sei \(A\subset \R^n\) offen. Der reguläre Rand von \(A\) ist die Menge \( \partial_r A \) 
    der Punkte \(x \in \partial A\), für die es \( p > 0 \) und \( G \in \mathcal{C}^1(B_p(x)) \) gibt, 
    sodass überall \( DG \neq 0 \) gilt sowie 
    \[ A \cap B_{p}(x) = G^{-1}((-\infty,0)). \]

    Innen ist \(G < 0\), außen ist \(G>0\) und auf dem Rand ist \(G = 0\).

    \(A\) heißt \(\mathcal{C}^1\)-berandet, falls \( \partial A = \partial_r A \).

    Es gilt \( \partial_r A \cap B_p (x) = G^{-1}(\set{0}) \).
\end{karte}

\begin{karte}{Äußerer Normalenvektor}
    Sei \( A\subset \R^n \) offen. Dann gibt es für jedes \( x\in \partial_r A \) einen Vektor 
    \( \nu(x) \in \R^n \) mit 
    \begin{enumerate}
        \item \(\abs{\nu(x)} = 1\).
        \item \( \nu(x) \bot T_x(\partial_r A), x \in \partial_r A \).
        \item \( \forall x \in \partial_r A \) gibt es \( \varphi > 0 \):
        \[ x + t\nu(x) \in A \;\forall -\varphi < t < 0 \qquad x + tc(x) \notin A \;\forall 0 < t < \varphi \]
    \end{enumerate}
    \( \nu(x) = \frac{\nabla G(x)}{ \abs{ \nabla G(x) }} \) erfüllt dies.
\end{karte}