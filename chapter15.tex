\section{Fast überall}

\begin{karte}{\(\mu\)-fast alle}
	Sei \(E\) eine Eigenschaft, die für jeden Punkt \(x\in X\) sinnvoll ist.\\
	Wir sagen \( \mu \)-fast alle Punkte \(x\in X\) besitzen die Eigenschaft \(E\) 
	oder \(E\) gilt \(\mu\)-fast überall, wenn es eine \(\mu\)-Nullmenge \(N\) gibt, 
	sodass alle \(x \in N^C\) die Eigenschaft \(E\) besitzen.
\end{karte}

\begin{karte}{Integral \(\mu\)-fast überall}
	Sei \(f \in \mathcal{M}^+\). Dann gilt \( \int f \dmu = 0 \Leftrightarrow f = 0 \) \(\mu\)-fast überall.
\end{karte}