\section*{Fast überall}

\begin{karte}{\(\mu\)-fast alle}
	Sei \(E\) eine Eigenschaft, die für jeden Punkt \(x\in X\) sinnvoll ist.\\
	Wir sagen \( \mu \)-fast alle Punkte \(x\in X\) besitzen die Eigenschaft \(E\) 
	oder \(E\) gilt \(\mu\)-fast überall, wenn es eine \(\mu\)-Nullmenge \(N\) gibt, 
	sodass alle \(x \in N^C\) die Eigenschaft \(E\) besitzen.
\end{karte}

\begin{karte}{Integral \(\mu\)-fast überall}
	Sei \(f \in \mathcal{M}^+\). Dann gilt \( \int f \dmu = 0 \Leftrightarrow f = 0 \) \(\mu\)-fast überall.
\end{karte}

\begin{karte}{Integral über \(\mu\)-Nullmenge}
	Jede \(\mathcal{A}\)-messbare Funktion \(f\) auf \(X\) ist 
	über einer beliebigen \( \mu \)-Nullmenge \(N\) 
	integrierbar und \( \int_N f \dmu = 0 \).
\end{karte}

\begin{karte}{\(\mu\)-fast überall gleiche Funktionen}
	Seien \( f,g\ \mathcal{A} \)-messbare numerische Funktionen 
	auf \(X\), welche \(\mu\)-fast überall gleich sind.
	Dann gilt 
	\begin{itemize}
		\item \( f \geq 0, g\geq 0 : \int f \dmu = \int g \dmu \).
		\item \(f\) ist integrierbar \( \Leftrightarrow\ g \) ist integrierbar und \( \int f \dmu = \int g \dmu \). Funktioniert auch, wenn \(f,g\) komplexwertig sind.
	\end{itemize}
\end{karte}

\begin{karte}{Majorante für Integrale}
	Seien \( \abb{f}{X}{\bar{\R}}, \abb{g}{X}{\bar{\R}} \) 
	und \(\abs{f} \leq g \ \mu\)-fast überall. 
	Dann ist mit \(g\) auch \(f\) integrierbar.
\end{karte}

\begin{karte}{Umgekehrte Monotonie des Integrals}
	Seien \( f,g \) integrierbare numerische Funktionen auf \(X\). \\
	Gilt \( \int_A f\dmu \leq \int_A g \dmu \;\forall A\in \mathcal{A} \Rightarrow f \leq g \ \mu \)-fast überall.\\
	Gilt \gqq{\(=\)}, so folgt auch \(f = g\ \mu\)-fast überall.
\end{karte}