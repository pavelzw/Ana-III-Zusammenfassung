\begin{karte}{Dynkin-System}
	Ein System von Teilmengen in \(X\) heißt 
	\textit{Dynkin-System} \(\mathcal{D} \subset \mathcal{P}(X)\), falls 
	\begin{enumerate}
		\item \(X \in \mathcal{D}\).
		\item \( A \in \mathcal{D} \Rightarrow A^C \in \mathcal{D} \).
		\item \( \forall (A_n)_n \subset \mathcal{D} \) disjunkt: \( \bigcup_{n=1}^\infty A_n \in \mathcal{D} \).
	\end{enumerate}
	\( \mathcal{D} \) ist stabil unter Bildung eigentlicher Komplemente, d. h. \( D,E \in \mathcal{D}, D \subset E \Rightarrow E \setminus D \in \mathcal{D} \).\\
	Ein Dynkin-System ist eine \(\sigma\)-Algebra genau 
	dann, wenn für je zwei Mengen in \(\mathcal{D}\) auch 
	deren Durchschnitt in \( \mathcal{D} \) liegt.
\end{karte}