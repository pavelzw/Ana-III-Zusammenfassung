\section*{Dynkin-Systeme}

\begin{karte}{Dynkin-System}
	Ein System von Teilmengen in \(X\) heißt 
	\textit{Dynkin-System} \(\mathcal{D} \subset \mathcal{P}(X)\), falls 
	\begin{enumerate}
		\item \(X \in \mathcal{D}\).
		\item \( A \in \mathcal{D} \Rightarrow A^C \in \mathcal{D} \).
		\item \( \forall (A_n)_n \subset \mathcal{D} \) disjunkt: \( \bigcup_{n=1}^\infty A_n \in \mathcal{D} \).
	\end{enumerate}
	\( \mathcal{D} \) ist stabil unter Bildung eigentlicher Komplemente, d. h. \( D,E \in \mathcal{D}, D \subset E \Rightarrow E \setminus D \in \mathcal{D} \).\\
	Ein Dynkin-System ist eine \(\sigma\)-Algebra genau 
	dann, wenn für je zwei Mengen in \(\mathcal{D}\) auch 
	deren Durchschnitt in \( \mathcal{D} \) liegt.
\end{karte}
\begin{karte}{Erzeugung eines Dynkin-Systems}
	Sei \( \mathcal{E} \subset \mathcal{P}(X) \). \(\delta(\mathcal{E})\) ist das kleinste Dynkin-System, 
	welches \( \mathcal{E} \) enthält. 
	\[ \delta(\mathcal{E}) = \bigcap_{\mathcal{D} \text{ Dynkin} \atop \mathcal{E} \subset \mathcal{D} } \mathcal{D}.  \]
	Also gilt immer \( \delta(\mathcal{E}) \subset \sigma(\mathcal{E}) \). 
\end{karte}
\begin{karte}{\( \cap\)-stabil, \(\cup\)-stabil, Lemma von Dynkin}
	Wir nennen \( \mathcal{E}\ \cap \)-stabil, falls 
	\( E, E_1 \in \mathcal{E} \Rightarrow E \cap E_1 \in \mathcal{E} \) und analog \(\cup\)-stabil.
	Ist \( \mathcal{E} \) ein \(\cap\)-stabiler Erzeuger, so ist \(\delta(\mathcal{E})\) eine \( \sigma \)-Algebra, 
	d. h. \( \delta(\mathcal{E}) = \sigma(\mathcal{E}) \).
\end{karte}