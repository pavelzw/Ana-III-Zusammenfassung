\section*{Produktmaße, Fubini und Tunelli}

\begin{karte}{Schnitt von Produkt-\(\sigma\)-Algebra}
    Gegeben zwei Messräume \( (X_j,\mathcal{A}_j) \).
    Für beliebige \( x_1 \in X_1, x_2 \in X_2 \) 
    liegen alle \(x_1\)- bzw. \(x_2\)-Schnitte einer Menge 
    (\( Q_{x_1} := \set{x_2 \in X_2 \;|\; (x_1,x_2)\in Q} \))
    \( W \in \mathcal{A}_1 \otimes \mathcal{A}_2 \) 
    in \(\mathcal{A}_2\) bzw. \( \mathcal{A}_1 \).
\end{karte}

\begin{karte}{Messbarkeit der Schnittfunktion}
    Die Maße \( \mu_1, \mu_2 \) seien \( \sigma \)-endlich. 
    So gilt \( \forall Q \in \mathcal{A}_1 \otimes \mathcal{A}_2 \) 
    ist die Funktion 
    \[ x_1 \mapsto \mu_2(Q_{x_1}) \ \mathcal{A}_1\text{-messbar}, \]
    \[ x_2 \mapsto \mu_1(Q_{x_2}) \ \mathcal{A}_2\text{-messbar}. \]
    Außerdem gilt dann 
    \( \pi_1(Q) := \int_{X_1} \mu_2(Q_{x_1}) \mu_1(\mathrm{d}x_1) \)
    und \\
    \( \pi_2(Q) := \int_{X_2} \mu_1(Q_{x_2}) \mu_2(\mathrm{d}x_2) \)
    sind nach Eindeutigkeit gleich.
\end{karte}

\begin{karte}{Prinzip von Cavalieri}
    Seien \( (X_j, \mathcal{A}_j, \mu_j) \) \(\sigma\)-endliche Maßräume, \(j=1,2\).
    
    Dann existiert genau ein Maß \(\pi\) auf \( \mathcal{A}_1 \otimes \mathcal{A}_2 \) 
    mit 
    \[ \pi(A_1 \times A_2) = \mu_1(A_1) \mu_2(A_2) \]
    \( \forall A_1\in \mathcal{A}_1, A_2 \in \mathcal{A}_2 \) und 
    für jede Menge \(Q \in \mathcal{A}_1 \otimes \mathcal{A}_2\) gilt:
    \[ \pi(Q) = \int_{X_1} \mu_2(Q_{x_1}) \mu_1(\mathrm{d}x_1) 
    = \int_{X_2} \mu_1(Q_{x_2}) \mu_2(\mathrm{d}x_2). \]
\end{karte}

\begin{karte}{Produktmaß}
    Das für je zwei \(\sigma\)-endliche Maßräume \( ( X_j, \mathrm{A}_j, \mu_j ) \)
    eindeutig bestimmte Maß \( \pi \) auf \( \mathcal{A}_1 \otimes \mathcal{A} \)
    heißt das Produkt der Maße \( \mu_1 \) und \( \mu_2 \). Wir schreiben \( \pi = \mu_1 \otimes \mu_2 \).
\end{karte}

\begin{karte}{\(x_i\)-Schnitt in Funktionen}
    Zu \( \abb{f}{X_1\times X_2}{Y} \) ist 
    \( \abb{f_{x_1}}{X_2}{Y}, x_2 \mapsto f_{x_1}(x_2) = f(x_1, x_2) \) der \( x_1 \)-Schnitt in \(f\).

    Analog \(x_2\)-Schnitt.

    Ist \( f = \mathds{1}_Q, Q \subset X_1 \times X_2 \), so ist 
    \( f_{x_1} = \mathds{1}_{Q_{x_1}} \) auf \(X_2\) und \( f_{x_2} = \mathds{1}_{Q_{x_2}} \) 
    auf \(X_1\).
\end{karte}

\begin{karte}{Messbarkeit \(x_i\)-Schnitt in Funktionen}
    Sei \( (X', \mathcal{A}') \) ein beliebiger Maßraum 
    und \( \abb{f}{(X_1\times X_2, \mathcal{A}_1 \otimes \mathcal{A}_2)}{(X', \mathcal{A}')} \)
    eine messbare Abbildung. Dann ist jede der Abbildungen \( f_{x_1} \) bzw. \( f_{x_2} \)
    \( \mathcal{A}_1 \)- bzw. \(\mathcal{A}_2\)-messbar.
\end{karte}

\begin{karte}{Satz von Tonelli}
    Seien \( (X_j, \mathcal{A}_j, \mu_j), j=1,2 \) \(\sigma\)-endliche Maßräume 
    und \( \abb{f}{X_1 \times X_2}{[0,\infty]} \) eine positive messbare 
    numerische Funktion. Dann gilt: die Funktionen \( x_2 \mapsto \int f_{x_2} \mathrm{d}\mu_1 \)
    und \( x_1 \mapsto \int f_{x_1} \mathrm{d}\mu_2 \) sind \( \mathcal{A}_2 \)- bzw. 
    \( \mathcal{A}_1 \)-messbar und es gilt 

    \begin{align*}
        \int_{X_1\times X_2} f \mathrm{d}(\mu_1\otimes\mu_2) &= \int \left( \int f_{x_2}\mathrm{d}\mu_1 \right) \mathrm{d}\mu_2 = \int_{X_2} \left( \int_{X_1} f(x_1, x_2) \mu_1(\mathrm{d}x_1) \right) \mu_2(\mathrm{d}x_2) \\
        &= \int \left(\int f_{x_1} \mathrm{d}\mu_2 \right) \mathrm{d}\mu_1 = \int_{X_1} \left( \int_{X_2} f(x_1, x_2) \mu_2(\mathrm{d}x_2) \right) \mu_1(\mathrm{d}x_1)
    \end{align*}
\end{karte}

\begin{karte}{Satz von Fubini}
    Seien \( (X_j, \mathcal{A}_j, \mu_j), j=1,2 \) \(\sigma\)-endliche 
    Maßräume und \(f\) eine \(\mu_1 \otimes \mu_2\) integrierbare numerische 
    Funktion auf \( X_1 \otimes X_2 \). Dann folgt 
    \begin{center}
        \( f_{x_1} \) ist für \( \mu_1 \)-fast alle \(x_1\) \(\mu_2\)-integrierbar\\
        \(f_{x_2}\) ist für \( \mu_2 \)-fast alle \(x_2\) \(\mu_1\)-integrierbar,
    \end{center}
    die Funktionen 
    \[ x_1 \mapsto \int_{X_2} f_{x_1} \dx{\mu_2}, \quad x_2\mapsto \int_{X_1} f_{x_2} \dx{\mu_1} \]
    sind \(\mu_1\)- bzw. \(\mu_2\)-fast überall definiert und es gilt 
    \begin{align*}
        \int_{X_1\times X_2} f \mathrm{d}(\mu_1\otimes\mu_2) &= \int \left( \int f_{x_2}\mathrm{d}\mu_1 \right) \mathrm{d}\mu_2 = \int_{X_2} \left( \int_{X_1} f(x_1, x_2) \mu_1(\mathrm{d}x_1) \right) \mu_2(\mathrm{d}x_2) \\
        &= \int \left(\int f_{x_1} \mathrm{d}\mu_2 \right) \mathrm{d}\mu_1 = \int_{X_1} \left( \int_{X_2} f(x_1, x_2) \mu_2(\mathrm{d}x_2) \right) \mu_1(\mathrm{d}x_1)
    \end{align*}
\end{karte}

\begin{karte}{Layer-Cake-Principle}
    Sei \( (X, \mathcal{A}, \mu) \) ein \(\sigma\)-endlicher Maßraum, 
    \( \abb{f}{[0,\infty)}{[0,\infty)} \) eine messbare 
    positive reelle Funktion \( \abb{\varphi}{[0,\infty)}{[0,\infty)}, \varphi(0)= 0 \)
    eine stetig wachsende Funktion, die stetig differenzierbar auf \((0,\infty)\) ist. 
    Dann folgt: 
    \[ \int \varphi \circ f \dmu = \int_{(0,\infty)} \varphi'(t) \mu(\set{f\geq t}) \;\lambda^1(\mathrm{d}t) 
    = \int_0^\infty \varphi'(t) \mu(\set{f\geq t}) \dx{t}. \]
\end{karte}