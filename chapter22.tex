\section*{Produktmaße, Fubini und Tunelli}

\begin{karte}{Schnitt von Produkt-\(\sigma\)-Algebra}
    Gegeben zwei Messräume \( (X_j,\mathcal{A}_j) \).
    Für beliebige \( x_1 \in X_1, x_2 \in X_2 \) 
    liegen alle \(x_1\)- bzw. \(x_2\)-Schnitte einer Menge 
    (\( Q_{x_1} := \set{x_2 \in X_2 \;|\; (x_1,x_2)\in Q} \))
    \( W \in \mathcal{A}_1 \times \mathcal{A}_2 \) 
    in \(\mathcal{A}_2\) bzw. \( \mathcal{A}_1 \).
\end{karte}

\begin{karte}{Messbarkeit der Schnittfunktion}
    Die Maße \( \mu_1, \mu_2 \) seien \( \sigma \)-endlich. 
    So gilt \( \forall Q \in \mathcal{A}_1 \otimes \mathcal{A}_2 \) 
    ist die Funktion 
    \[ x_1 \mapsto \mu_2(Q_{x_1}) \ \mathcal{A}_1\text{-messbar}, \]
    \[ x_2 \mapsto \mu_1(Q_{x_2}) \ \mathcal{A}_2\text{-messbar}. \]
    Außerdem gilt dann 
    \( \pi_1(Q) := \int_{X_1} \mu_2(Q_{x_1}) \mu_1(\mathrm{d}x_1) \)
    und \\
    \( \pi_1(Q) := \int_{X_2} \mu_1(Q_{x_2}) \mu_2(\mathrm{d}x_2) \)
    sind nach Eindeutigkeit gleich.
\end{karte}