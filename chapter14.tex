\section*{Integrierbare Funktionen}

\begin{karte}{\( \mu \)-integrierbar}
	Sei \( (X, \mathcal{A}, \mu) \) ein Maßraum. Eine numerische Funktion 
	\( \abb{f}{X}{\bar{\R}} \) heißt \(\mu\)-integrierbar, falls sie 
	\( \mathcal{A} \)-messbar ist und die Integrale \( \int f^- \dmu, \int f^+ \dmu \) 
	reelle Zahlen sind. Dann heißt 
	\[ \int f \dmu := \int f^+ \dmu - \int f^- \dmu \]
	das \( \mu \)-Integral von \( f \) auf \(X\).\\
	Die rechte Seite ist wohldefiniert, falls mind. einer der Terme \( \int f^+ \dmu \) oder 
	\( \int f^- \dmu \) endlich sind. Man sagt dann, \(f\) ist quasi integrierbar oder 
	das Integral von \(f\) existiert.
	Wir schreiben auch 
	\( \int f(x) \dmu(x) \) oder \( \int f(x) \mu(\mathrm{d}x) \). \\
	Ist \( f \geq 0, f^- \equiv 0 \Rightarrow \int f \dmu\) ist konstant mit dem Integral von 
	\( f\in \mathcal{M}^+ \).
\end{karte}

\begin{karte}{Bedingungen für Integrierbarkeit}
	Notwendig und hinreichend für die Integrierbarkeit einer messbaren, numerischen Funktion 
	\( \abb{f}{X}{\bar{\R}} \) ist jede der folgenden Bedingungen 
	\begin{itemize}
		\item \( f^+ \) und \(f^-\) sind integrierbar.
		\item \(\exists\) integrierbare Funktionen \(u, v \geq 0\) mit \(f = u - v\).\\
		Es gilt dann \( \int f \dmu = \int u \dmu - \int v \dmu \).
		\item \( \exists \) integrierbare Funktion \( g\geq 0 \) mit \( \abs{f} \leq g \).
		\item \( \abs{f} \) ist integrierbar.
	\end{itemize}
\end{karte}

\begin{karte}{Linearität des Integrals}
	Seien \( \abb{f,g}{X}{\bar{\R}} \) integrierbare, numerische 
	Funktionen. \( \Rightarrow \alpha f, \alpha \in \R \) 
	und \( f + g \) (sofern überall definiert) sind integrierbar 
	und es gilt 
	\[ \int \alpha f \dmu = \alpha \int f \dmu, \quad
	\int f+g \dmu = \int f \dmu + \int g \dmu. \]
	Ferner sind \(f \vee g\) und \( f \wedge g \) integrierbar.
\end{karte}

\begin{karte}{Monotonie des Integrals}
	Seien \( \abb{f,g}{X}{\bar{\R}} \) integrierbare, 
	numerische Funktionen und es gelte \( f \leq g \).
	\[ \Rightarrow \int f \dmu \leq \int g \dmu, \quad 
	\abs{\int f \dmu} \leq \int \abs{f}\dmu. \]
\end{karte}

\begin{karte}{Integral von unendlichen Funktionen}
	Für jede \( \mu \)-integrierbare numerische Funktion \( f \) auf \(X\) ist 
	\( \set{\abs{f} = +\infty} \) eine \(\mu\)-Nullmenge.
\end{karte}

\begin{karte}{Integrale über \( \C\)}
	\( \abb{f}{X}{\C} \) heißt \( \mu \)-integrierbar über \( X \), wenn \( f \) messbar ist und 
	\( \Re(f), \Im(f) \) integrierbar sind. \\
	Wir setzen dann \( \int f \dmu := \int \Re(f) \dmu + i \cdot \int \Im(f) \dmu \).\\
	Seien \( \abb{f,g}{X}{\C} \) integrierbar, \(\alpha \in \C\). Dann gilt 
	\[ \int \alpha f \dmu = \alpha \int f\dmu, \quad 
	\int f + g \dmu = \int f \dmu + \int g \dmu. \]
	Ist \(\abb{f}{X}{\C}\) integrierbar, so gilt 
	\[ \abs{ \int f \dmu } \leq \int \abs{f} \dmu. \]
	Seien \( \abb{f,g}{X}{\C} \) quasi-integrierbar und \( f \leq g \). 
	Dann gilt \( \int f \dmu \leq \int g \dmu \).
\end{karte}

\begin{karte}{Integrierbarkeitsbedingungen über \( \C\)}
	Für jede Funktion \( \abb{f}{X}{\C} \) sind äquivalent 
	\begin{itemize}
		\item \(f\) ist integrierbar.
		\item \( \Re f \) und \(\Im f\) sind integrierbar.
		\item \( (\Re f)^\pm \) und \( (\Im f)^\pm \) sind integrierbar.
		\item \( \exists \) reelle integrierbare Funktionen 
		\(p,q,r,s \geq 0 \) mit \( f = p - q + i \cdot (r - s) \).
		\item \( f \) ist messbar und \( \exists \) integrierbare Funktion \( g \geq 0: \abs{f}\leq g \).
		\item \( f \) ist messbar und \( \abs{f} \) integrierbar.
	\end{itemize}
\end{karte}

\begin{karte}{Konvexe Funktionen}
	Eine Funktion \( \abb{f}{\R^d}{\R} \) heißt \textit{konvex}, 
	falls für \( x,y\in \R^d \) und \( \theta \in [0,1] \) gilt, dass 
	\[ f(\theta x + (1 - \theta)y) \leq \theta f(x) + (1 - \theta) f(y). \]
	Äquivalent sind 
	\begin{itemize}
		\item \(f\) ist konvex.
		\item \( f(t) \leq f(x) + \frac{f(y) - f(x)}{y - x} (t-x), \;x < t<y \).
		\item \( \frac{f(t) - f(x)}{t - x} \leq \frac{f(y) - f(x)}{y - x} \).
		\item \( \frac{f(t) - f(x)}{t - x} \leq \frac{f(y) - f(t)}{y - t}, \; x < t < y \).
	\end{itemize}
\end{karte}

\begin{karte}{Jensensche Ungleichung}
	Sei \( (X, \mathcal{A}, \mu) \) ein Maßraum mit 
	\( \mu(X) = 1, I \subset \R \) ein Intervall und 
	\( \abb{f}{X}{I} \) \(\mu\)-integrierbar sowie 
	\( \abb{\phi}{I}{\R} \) konvex, dann gilt 
	\[ m := \int f \dmu \in I, \phi \circ f \text{ ist integrierbar und } \]
	\[ \phi\left(\int f\dmu \right) \leq \int \phi \circ f \dmu. \]
\end{karte}

\begin{karte}{Integral über \(A \in \mathcal{A}\)}
	Sei \(\abb{f}{X}{\bar{\R}}\) integrierbar oder in \( \mathcal{M}^+ \), \(A \in \mathcal{A}\). \\
	Wir setzen 
	\[ \int_A f \dmu := \int \mathds{1}_A f \dmu \] 
	und nennen dies das \( \mu \)-Integral über \(A\).\\
	Insbesondere gilt \( \int_X f\dmu = \int f \dmu \).
	\begin{itemize}
		\item \( \int_{A\cup B} f\dmu + \int_{A\cap B} f\dmu = \int_A f \dmu + \int_B f \dmu \).
		\item \( \int_{A\cup B} f\dmu = \int_A f \dmu + \int_B f \dmu \), falls \(A \cap B = \emptyset\).
		\item \( f \leq g \) auf \(A \Rightarrow \int_A f \dmu \leq \int_A g \dmu\).
	\end{itemize}
\end{karte}

\begin{karte}{\(\mathcal{L}^1(\mu)\)}
	Wir setzen \( \mathcal{L}^1(\mu) := \) Menge aller reellwertigen \( \mu \)-integrierbaren Funktionen auf \(X\).\\
	Wir setzen \( \mathcal{L}^1_\C(\mu) := \) Menge aller komplexwertigen \( \mu \)-integrierbaren Funktionen auf \(X\).\\
	Es gilt \( \int f \dx{(\mu + \nu)} = \int f \dmu + \int f \dx{\nu} \).\\
	\( \mathcal{L}^1(\mu) \) ist ein Vektorraum.\\
	\(f \mapsto \int_A f \dmu\) ist eine monotone Linearform auf \( \mathcal{L}^1(\mu)  \) für \(A\in \mathcal{A}\).
\end{karte}

\begin{karte}{Restriktion einer Funktion}
	Sei \(f \in \mathcal{L}^1(\mu)\). \(f'\) ist die Restriktion von \(f\) auf \(A\).\\
	\( \Rightarrow f' \in \mathcal{L}^1(\mu_A)\) und \( \int f' \dx{\mu_A} = \int_A f \dmu \) für \( \mu_A := \mu_{A \cap \mathcal{A}} \).
\end{karte}

\begin{karte}{Erweiterung einer Funktion auf \(X\)}
	Sei \( f \) eine auf einer Menge \(A \in \mathcal{A}\) definierte numerische Funktion. Dann gilt \\
	\(f\) ist über \( A \) \(\mu\)-integrierbar \( \Leftrightarrow \) 
	die durch \( f_A(x) := \begin{cases}
	f(x) & x\in A\\ 0 & x \notin A
	\end{cases} \) auf ganz \(X\) definierte Funktion ist \(\mu\)-integrierbar.\\
	Es gilt \( \int_A f \dmu = \int f_A \dmu = \int \mathds{1}_A f \dmu \).
\end{karte}