\section*{Elementarfunkt. und ihr Integral}

\begin{karte}{\( \mathcal{A} \)-Elementarfunktion}
	Sei \( \mathcal{A} \) eine \(\sigma\)-Algebra in \(X\). 
	Eine Funktion \( \abb{f}{X}{\R} \) heißt \(\mathcal{A}\)-Elementarfunktion, 
	wenn sie nur endlich viele Werte annimmt und \(\mathcal{A}\)-messbar ist.\\
	Wir schreiben \( E = E(X, \mathcal{A}) \) als die Menge aller Elementarfunktionen, 
	\( E^+ = \set{ f\in E \;|\; f \geq 0 } \) die Menge aller positiven Elementarfunktionen.
\end{karte}

\begin{karte}{Normaldarstellung}
	Jedes \( u \in E \) hat Darstellung \( u = \sum_{j=1}^n a_j \mathds{1}_{A_j} \) 
	mit \( A_1,\ldots, A_n \) paarweise disjunkt und \( X = \bigcup_{j=1}^n A_j \), also \( A_j \) Zerlegung von \(X\).
	Wir nennen eine solche Darstellung von \(u\) \textit{Normaldarstellung}.
\end{karte}

\begin{karte}{Maß von Normaldarstellungen}
	Sei \( (X, \mathcal{A}, \mu) \) ein Maßraum. Für je zwei Normaldarstellungen 
	\( u = \sum_{j=1}^n \alpha_j \mathds{1}_{A_j} = \sum_{l=1}^m \beta_l \mathds{1}_{B_l} \) einer 
	positiven Elementarfunktion \( u \in E^+ \) gilt: 
	\[ \sum_{j=1}^n \alpha_j \mu(A_j) 
	= \sum_{l=1}^m \beta_l \mu(B_l). \]
\end{karte}

\begin{karte}{\( \mu \)-Integral}
	Sei \( u \in E^+ \). Dann heißt die von der speziellen Normaldarstellung 
	\( u = \sum_{j=1}^n \alpha_j \mathds{1}_{A_j} \) unabhängige 
	Zahl 
	\[ \int u \dx{\mu} 
	:= \sum_{j=1}^n \alpha_j \mu(A_j) \]
	das \( \mu \)-Integral von \(u\) über \(X\).\\
	\( u \mapsto \int u \dx{\mu} \) definiert eine Abbildung von 
	\(E^+\) nach \( \bar{\R}_+ \) oder sogar nach \( \R_+ \), wenn \(\mu\) 
	endlich ist.\\
	Wichtige Eigenschaften:
	\begin{enumerate}
		\item \( \int \mathds{1}_A \dx{\mu} = \mu(A) \).
		\item \( \int \alpha u \dx{\mu} = \alpha \int u \dx{\mu} \) für \( \alpha \geq 0, u \in E^+ \).
		\item \( \int u + v \dx{\mu} = \int u \dx{\mu} + \int v \dx{\mu} \) für \( u,v \in E^+ \).
		\item \( \int u \dx{\mu} \leq \int v \dx{\mu} \) für \( u \leq v \in E^+ \).
	\end{enumerate}
\end{karte}