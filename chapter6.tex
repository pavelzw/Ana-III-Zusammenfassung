\section*{Maßerweiterung von Carathéodory}

\begin{karte}{Äußeres Maß}
	Sei \(A \subset X\) und \( \mathcal{H} \subset \mathcal{P}(X) \) ein Halbring.
	Weiter sei für ein beliebiges \(A \subset X\) 
	\[ J(A) := \set{ (A_n)_n \subset \mathcal{H} \;|\; A \subset \bigcup_{n=1}^\infty A_n }. \]
	Falls keine Überdeckung existiert, ist \(J(A) = \emptyset\).\\
	Ein äußeres Maß \( \mu_* \) zu \(\mu\) ist 
	\[ \mu_*(A) = \inf \set{ \sum_{n=1}^\infty \mu(A_n) \;|\; (A_n)_n\in J(A) }, \]
	falls es mindestens eine solche Überdeckung gibt.
	Andernfalls ist \( \mu_*(A) := \infty \).
\end{karte}

\begin{karte}{Äußeres Maß nach Carathéodory}
	Ein äußeres Maß ist eine Abbildung \( \abb{\eta}{\mathcal{P}(X)}{[0,\infty]} \), für welche gilt:
	\begin{enumerate}
		\item \(\eta(\emptyset) = 0\)
		\item \( \forall A \subset B \subset X: \eta(A) \leq \eta(B) \) (Monotonie)
		\item \( \forall (A_n)_n, A_n \subset X: \eta(\bigcup_{n=1}^\infty A_n) \leq \sum_{n=1}^\infty \eta(A_n) \) (\(\sigma\)-Subadditivität)
	\end{enumerate}
\end{karte}

\begin{karte}{\(\eta\)-messbar}
	Sei \( \abb{\eta}{\mathcal{P}(X)}{[0,\infty]} \) ein äußeres Maß. Dann heißt 
	\( A\subset X \) \(\eta\)-messbar, falls für alle \( Q \subset X \) gilt 
	\[ \eta(Q) \geq \eta(Q\cap A) + \eta(Q \cap A^C) \]
\end{karte}

\begin{karte}{Eigenschaften \(\eta\)-messbar}
	Sei \( \abb{\eta}{\mathcal{P}(X)}{[0,\infty]} \) ein äußeres Maß, \(A\subset X\). Dann gilt 
	\begin{enumerate}
		\item Ist \( \eta(A) = 0 \) oder \( \eta(A^C) = 0 \), dann ist \(A\) \(\eta\)-messbar.
		\item \( A \) ist \(\eta\)-messbar \( \Leftrightarrow \forall Q\subset X \) mit 
		\( \eta(Q) \geq \eta(Q\cap A) + \eta(Q \cap A^C) \).
		\item \( A \) ist \(\eta\)-messbar \( \Leftrightarrow \forall Q\subset X: 
		\eta(Q) = \eta(Q\cap A) + \eta(Q\cap A^C) \) zerlegt in zwei disjunkte Teilmengen.
	\end{enumerate}
\end{karte}

\begin{karte}{Satz von Carathéodory}
	Ist \( \abb{\eta}{\mathcal{P}(X)}{[0,\infty]} \) ein äußeres Maß, so ist 
	\[ \mathcal{A}_\eta := \set{ A\subset X \;|\; \text{A ist \(\eta\)-messbar} } \]
	eine \(\sigma\)-Algebra und \(\eta\) ein Maß.
\end{karte}

\begin{karte}{Fortsetzungssatz}
	Sei \( \abb{\mu}{\mathcal{H}}{[0,\infty]} \) ein Inhalt auf einem 
	Halbring \( \mathcal{H} \subset \mathcal{P}(X) \) und für \( A \subset X \) 
	gilt 
	\[ \eta(A) := \inf \set{ \sum_{n=1}^\infty \mu(A_n) \;|\; A_n \in \mathcal{H}, A \subset \bigcup_{n=1}^\infty A_n }, 
	\qquad \inf(\emptyset) := \infty, \]
	so gilt 
	\begin{enumerate}
		\item \( \abb{\eta}{\mathcal{P}(X)}{[0,\infty]} \) ist ein äußeres Maß 
		und alle Mengen aus \( \mathcal{H} \) sind \(\eta\)-messbar.
		\item Ist \(\mu\) ein Prämaß, so gilt \( \eta|_{\mathcal{H}} = \mu \) 
		und somit ist \( \eta|_{\mathcal{A}_\eta} \) eine Fortsetzung von \(\mu\) 
		zu einem Maß auf der \( \sigma \)-Algebra \( \mathcal{A}_\eta \), die 
		\( \mathcal{H} \) und somit auch \( \sigma(\mathcal{H}) \) enthält.
		\item Ist \(\mu\) kein Prämaß, so gibt es \( A\in \mathcal{H} \) 
		mit \( \eta(A) \leq \mu(A)\).
	\end{enumerate}
\end{karte}


\begin{karte}{Verteilungsfunktion}
	Sei \( \abb{F}{\R}{\R} \) rechtsseitig stetig und monoton wachsend, dann 
	heißt \(F\) \textit{Verteilungsfunktion}, falls \( \limesx{x}{-\infty}F(x) = 0 \) 
	und \( \limesx{x}{\infty}F(x) = 1 \).
\end{karte}

\begin{karte}{Lebesgue-Stieltjes-Maß}
	Für eine Verteilungsfunktion \(F\) heißt \( \mu_F((a,b]) = F(b) - F(a) \) Lebesgue-Stieltjes-Maß. 
	Falls \(\mu\) endlich und \(\mu(X) = 1\), dann ist
	\( F_\mu(x) := \mu((-\infty,x]) \) eine Verteilungsfunktion.
\end{karte}

\begin{karte}{Eigenschaften Stieltjes'scher Inhalt}
	Sei \( \mathcal{G} := \set{ (a,b] \;|\; a,b\in \R, a\leq b } \).
	Es gilt 
	\begin{enumerate}
		\item \( \abb{F}{\R}{\R} \) ist monoton wachsend und beschränkt 
		\( \Rightarrow \abb{\mu_f}{\mathcal{G}}{\R} \) mit 
		\( \mu_F((a,b]) = F(b) - F(a) \) für \(a\leq b\) ist ein Inhalt.
		\item Für \( \abb{F,G}{\R}{\R} \) gilt \( \mu_F = \mu_G \Leftrightarrow F - G\) ist konstant.
		\item Sei \( \abb{F}{\R}{\R} \) monoton wachsend, \( \abb{\mu_F}{\mathcal{G}}{\R} \) der Stieltjes'sche Inhalt. Dann gilt 
		\[ \mu_F \text{ ist ein Prämaß} \Leftrightarrow F \text{ ist rechtsseitig stetig.} \]
	\end{enumerate}
\end{karte}