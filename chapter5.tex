\section*{Inhalte, Ringe und Halbringe}

\begin{karte}{Ring}
	Sei \(X\) eine Menge. Ein System \(\mathcal{R} \subset \mathcal{P}(X) \) heißt \textit{Ring}, falls 
	\begin{enumerate}
		\item \( \emptyset \in \mathcal{R} \)
		\item \( A,B \in \mathcal{R} \Rightarrow A\setminus B \in \mathcal{R} \)
		\item \( A,B \in \mathcal{R} \Rightarrow A \cup B \in \mathcal{R} \).
	\end{enumerate}
	Fordert man zusätzlich noch \(X \in \mathcal{R}\), dann ist \(\mathcal{R}\) eine Algebra.
\end{karte}
