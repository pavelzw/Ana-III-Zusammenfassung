\section*{Das Integral pos. messb. num. Fkt.}

\begin{karte}{Integral vom Supremum von Funktionenfolgen}
	Sei \( (u_n)_n \subset E^+ \) eine wachsende Funktionenfolge 
	(\(u_n \leq u_{n+1}\)) und \(u \in E^+\).\\
	\( u \leq \sup_{n\in\N} u_n \Rightarrow \int u \dmu 
	\leq \sup_{n\in\N} \int u_n \dmu \).\\
	Sind \( (u_n)_n, (v_n)_n \) wachsende Folgen in \(E^+\). Dann gilt 
	\[ \sup_{n\in\N} u_n = \sup_{n\in\N} v_n \Rightarrow 
	\sup_{n\in\N} \int u_n \dmu = \sup_{n\in\N} \int v_n \dmu. \]
\end{karte}

\begin{karte}{\( \mathcal{M}^+ \)}
	\( \mathcal{M}^+ := \mathcal{M}^+(X, \mathcal{A}) 
	= \set{f \;|\; f \geq 0, \text{ es gibt eine wachsende Folge } (u_n)_n \subset E^+ 
	\text{ mit } f = \sup_{n\in\N} u_n. } \). Also alle numerischen Funktionen, für die es eine wachsende Folge im Supremum gibt.
	Es gilt für \( \sup_{n\in\N} u_n = f\in \mathcal{M}^+: \int f \dmu = \sup_{n\in\N} \int u_n \dmu \).\\
	Alle messbaren numerischen Funktionen \( f \geq 0 \) auf \(X\) sind in \( \mathcal{M}^+ \).
\end{karte}

\begin{karte}{Eigenschaften von \( \mathcal{M}^+ \)}
	Es gilt \( f,g \in \mathcal{M}^+ \Rightarrow \)
	\begin{itemize}
		\item \( \alpha f \in \mathcal{M}^+ \;\forall \alpha \in\R \).
		\item \( f + g \in \mathcal{M}^+ \).
		\item \( f \cdot g \in \mathcal{M}^+ \).
		\item \( f \vee g \in \mathcal{M}^+ \).
		\item \( f \wedge g \in \mathcal{M}^+ \).
		\item \( \int \alpha f \dmu 
		= \alpha \int f \dmu \).
		\item \( \int f+g \dmu 
		= \int f \dmu + \int g \dmu \).
		\item \( f \leq g \Rightarrow 
		\int f \dmu \leq \int g\dmu \).
	\end{itemize}
\end{karte}

\begin{karte}{Integral von messbaren Funktionen}
	Sei \( \abb{f}{X}{\R} \) messbar. Dann existiert eine monoton steigende 
	Folge (\( \phi_n\))von einfachen Funktionen, die gegen \(f\) konvergiert.
	Das Integral von \(f\) ist wie folgt definiert.
	\[ \int_X f \dmu = \limes{n} \int \phi_n \dmu. \]
\end{karte}

\begin{karte}{Satz von der monotonen Konvergenz (Beppo Levi)}
	Sei \( (f_n)_n \subset \mathcal{M}^+ \) eine wachsende Folge
	\( \Rightarrow \sup_{n\in\N} f_n \in \mathcal{M} \) und 
	\( \int \sup_{n\in\N} f_n \dmu 
	= \sup_{n\in\N} \int f_n \dmu \).\\
	Alternative Formulierung: 
	\( \int \limes{n} f_n \dmu 
	= \limes{n} \int f_n \dmu \).
\end{karte}

\begin{karte}{Integral von Funktionsreihen}
	\( \forall \) Folgen \( (f_n)_n \subset \mathcal{M}^+ \) gilt: 
	\[ \sum_{n=1}^{\infty} f_n \in \mathcal{M}^+ \text{ und } \int \sum_{n\in\N} f_n \dmu 
	= \sum_{n\in\N} \int f_n \dmu. \]
\end{karte}

\begin{karte}{Lemma von Fatou}
	Sei \( (f_n)_n \subset \mathcal{M}^+ \). Dann gilt 
	\[ \limesinf{n} f_n \in \mathcal{M}^+ 
	\text{ und } \int \limesinf{n} f_n \dmu \leq \limesinf{n} \int f_n \dmu. \]
\end{karte}

\begin{karte}{\(\limesinf{n}\) und \( \limessup{n} \) von Mengenfolgen}
	\( \forall (A_n)_n \subset \mathcal{A} \) gilt 
	\[ \mu(\limesinf{n} A_n) \leq \limesinf{n} \mu(A_n). \]
	Für \( \mu \) endlich gilt: 
	\[ \mu(\limessup{n} A_n) \geq \limessup{n} \mu(A_n). \]
\end{karte}