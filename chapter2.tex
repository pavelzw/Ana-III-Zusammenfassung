\section*{Mengentheoretische Grundlagen}

\begin{karte}{Eigenschaften injektiver Funktionen}
    \begin{align*}
    	&\abb{f}{X}{Y} \text{ ist injektiv } \\
    	&\Leftrightarrow f(A \cap B) = f(A) \cap f(B) \;\forall A, B \subset X \\
	   	&\Leftrightarrow f(X \setminus A) = f(X) \setminus f(A).
    \end{align*}
\end{karte}

\begin{karte}{Limes superior/inferior bei Mengen}
	Ist \((A_n)_n \subset \mathcal{P}(X) \), so heißt 
	\begin{itemize}
		\item \( \limessup{n} A_n := \lowlimes{n} A_n := \set{ x \in X: x \in A_n 
			\text{ für unendlich viele } n\in\N } \).
		\item \( \limesinf{n} A_n := \uplimes{n} A_n := \set{ x \in X: \exists N_0(x): x \in A_n 
			\forall n \geq N_0(x) } \).
	\end{itemize}
	Es gilt
	\[ \uplimes{n} A_n = \bigcap_{n\in\N} \bigcup_{k \geq n}A_k, 
	\qquad \lowlimes{n} A_n = \bigcup_{n\in\N} \bigcap_{k\geq n} A_k. \]
	Es gilt außerdem immer \(\lowlimes{n} A_n \subset \uplimes{n} A_n \).
\end{karte}

\begin{karte}{Konvergenz Mengenfolgen}
	\(A_n\) konvergiert, falls \(\uplimes{n} A_n = \lowlimes{n} A_n\). \\
	In diesem Fall nennen wir \(A := \limes{n} A_n = \uplimes{n} A_n = \lowlimes{n} A_n \) Grenzwert.\\
	Wir nennen \((A_n)_n\) 
	\begin{description}
		\item[wachsend,] wenn \( A_n \subset A_{n+1} \;\forall n \in \N \).
		\item[fallend,] wenn \( A_{n+1} \subset A_n \;\forall n \in \N \).
	\end{description}
\end{karte}

\begin{karte}{Monotone Konvergenz von Mengen}
	Jede monotone Folge \( (A_n)_n \subset X \) konvergiert. Es gilt 
	\begin{itemize}
		\item \( \limes{n} A_n = \bigcup_{n\in\N} A_n \), falls \( (A_n)_n \) wachsend ist.
		\item \( \limes{n} A_n = \bigcap_{n\in\N} A_n \), falls \( (A_n)_n \) fallend ist.
	\end{itemize}
\end{karte}